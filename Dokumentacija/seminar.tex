\documentclass[times, utf8, seminar, numeric]{fer}
\usepackage{booktabs, url, hyperref}
\usepackage{verbatim}
\usepackage{moreverb}
\usepackage{subfigure}
\usepackage{caption}
\usepackage{epstopdf}

\hypersetup{
   colorlinks,
   citecolor=black,
   filecolor=black,
   linkcolor=black,
   urlcolor=black
}

%dodatak za programski kod
\usepackage{listings}
\usepackage{color}
\usepackage{setspace}
\definecolor{dkgreen}{rgb}{0,0.6,0}
\definecolor{gray}{rgb}{0.5,0.5,0.5}
\definecolor{mauve}{rgb}{0.58,0,0.82}

\lstset{frame=tb,
  language=Java,
  aboveskip=3mm,
  belowskip=3mm,
  showstringspaces=false,
  columns=flexible,
  basicstyle={\small\ttfamily},
  numbers=left,
  numberstyle=\small\color{gray},
  keywordstyle=\color{blue},
  commentstyle=\color{dkgreen},
  stringstyle=\color{mauve},
  breaklines=true,
  breakatwhitespace=true,
  tabsize=2
}


\begin{document}

\nocite{saitou}\nocite{phylpro}\nocite{wiki}\nocite{mbt}

% TODO: Navedite naslov rada.
\title{Algoritam "Neighbor joining"}

% TODO: Navedite vaše ime i prezime.
\author{\itshape{Filip Beć}\\
				 \itshape{Zorana Ćurković}\\
				 \itshape{Goran Gašić}\\
				 \itshape{Melita Kokot}\\
				 \itshape{Dino Šantl}\\
				 \itshape{Igor Smolkovič}
				 }

% TODO: Navedite ime i prezime mentora.
\voditelj{\itshape{Dr. sc. Mirjana Domazet-Lošo}\\ \itshape{Doc. dr. sc. Mile Šikić}}

\maketitle

\tableofcontents

\chapter{Opis algoritma}

\chapter{Primjer izvođenja}

\chapter{Testiranje i usporedbe}

\chapter{Zaključak}

\bibliography{literatura}
%\bibliographystyle{fer} %promijena za citiranje po redu ieeetr
\bibliographystyle{ieeetr}

\begin{comment}
\begin{sazetak}
Simbolička regresija je postupak otkrivanja matematičkog izraza u skupu podataka. Daje se pregled metoda za simboličku regresiju s naglaskom na genetsko programiranje. Obrađuju se problemi kao što su domene funkcija (nisu definirane na cijelom skupu realnih brojeva). Problemi se rješavaju intervalnom aritmetikom i linearnim skaliranjem. Na kraju se ukratko opisuje mogućnost paralelizacije i primjene. 

\kljucnerijeci{genetsko programiranje, s}

\end{sazetak}

% TODO: Navedite naslov na engleskom jeziku.
\engtitle{Application of graphics coprocessors for program execution on stream programming model}

\begin{abstract}


\keywords{GPU, StreamIt, Sponge, StreamGate, CUDA, stream model, filter, optimization, graphics card}
\end{abstract}
\end{comment}

\end{document}
